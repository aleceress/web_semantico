% scopiazzato dal template di Matteo Longeri (grazie!)
%%%%%%%%%%%%%%%%%%%%%%%%%%%%%%%%%%%%%%%%%%%%%%%%%%%%%%
\documentclass[12pt,a4paper]{report}
% o article, book, ...



%%%%%%%%%%%%%%%%%%%%%%%%%%%%%%%%%%%%%%%%%%%%%%%%%%%%%%
% packages...
\usepackage[utf8]{inputenc}
\usepackage[english,italian]{babel}
\usepackage[hyphens]{url}

% Per generare il file PDF aderente alle specifiche PDF/A-1b. Verificarne poi la validità.
%\usepackage[a-1b]{pdfx}

\usepackage{hyperref}
\usepackage{graphicx}


% Per inserire testo a caso in attesa di realizzare i capitoli
\usepackage{lipsum}



%%%%%%%%%%%%%%%%%%%%%%%%%%%%%%%%%%%%%%%%%%%%%%%%%%%%%
\begin{document}

% Frontespizio
\begin{titlepage}
\begin{center}
\includegraphics[width=\textwidth]{Logo.jpg}\\
{\large{\bf Corso di Laurea in Informatica}}
\end{center}
\vspace{12mm}
\begin{center}
{\huge{\bf Apprendimento di insiemi fuzzy nell'ambito del web semantico}}\\
\end{center}
\vspace{12mm}
\begin{flushleft}
{\large{\bf Relatore:}}
{\large{Prof. Dario Malchiodi}}\\
\vspace{4mm}
{\large{\bf Correlatore:}}
{\large{Prof.ssa Anna Maria Zanaboni}}\\
\end{flushleft}
\vspace{12mm}
\begin{flushright}
{\large{\bf Tesi di Laurea di:}}
{\large{Alessia Cecere}}\\
{\large{\bf Matr. 923563}}\\
\end{flushright}
\vspace{4mm}
\begin{center}
{\large{\bf Anno Accademico 2020/2021}}
\end{center}
\end{titlepage}


\tableofcontents


% o sections (dipende dal documentclass)
\chapter{Introduzione}

\section{L'algoritmo di apprendimento}
\section{La ricerca di assiomi in un insieme di formule}

\chapter{Elementi del problema}

\section{Adattamento dell'algoritmo di apprendimento}
Qui indicherei come l'algoritmo di apprendimento viene adattato per essere applicato alla valutazione di formule.
\subsection{Utilizzo della libreria mulearn}
Qui descriverei mulearn e in particolare come è stata utilizzata durante gli esperimenti.
\section{Il Kernel}
Qui inizierei a descrivere il ruolo del kernel nella computazione.
\subsection{Kernel precomputato}
Qui descriverei il kernel precomputato utilizzato, passando per la matrice di Gram e per il relativo adjustment. Sempre in questo contesto spiegherei i problemi riscontrati durante il fitting, e la possibilità che siano stati causati proprio dal fatto che non sia esattamente un kernel.
\subsection{Alternative al kernel precomputato}
Qui descriverei le altre forme di computazione del kernel che sono state considerate.
\subsubsection{Length-based similarity}
\subsubsection{Hamming similarity}
\subsubsection{Levenshtein similarity}
\subsubsection{Jaccard similarity}

\chapter{Esperimenti}
Qui andrei a mostrare gli esperimenti e i relativi risultati conseguiti tramite cross validation e model selection.
\section{Riproduzione degli esperimenti originali}
\section{Esperimenti sul kernel}
\subsection{Kernel alternativi}
\subsection{Possibili soluzioni al problema del fitting}
\subsubsection{Eliminazione combinatoria di formule}
\subsubsection{Eliminazione a campione di formule}
\subsubsection{Valori di similarità come vettori in input all'algoritmo}


%\addcontentsline{toc}{chapter}{Bibliografia}

\end{document}
